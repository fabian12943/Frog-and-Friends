\setlength{\parindent}{0em}

\chapter{Spieleanleitung}

\label{Chapter2}

\section{Menü}
\label{menu}
Das Hauptmenü bietet dem Spieler folgende Auswahlmöglichkeiten:

\begin{enumerate}
    \item \textbf{Play}:\\
    Um eine Runde zu spielen, klickt man auf diesen Button. Anschließend muss man noch das Level auswählen, sowie die Anzahl der Spieler die mitspielen möchten. Daraufhin startet das Spiel.
    \item \textbf{Settings}:\\
    Um die Spieleinstellungen zu ändern, klickt man auf diesen Button. Es öffnet sich ein weiteres Untermenü, in welchem man mit zwei Schiebereglern die Gesamtlautstärke des Spiels einstellen kann, sowie eine Rundenlänge zwischen 30 Sekunden und 5 Minuten festlegen kann.
    \item \textbf{Exit}:\\
    Um das Spiel zu beenden, klickt man auf diesen Button.
\end{enumerate}

Um aus den Untermenüs der Optionen \textit{Play} und \textit{Settings} jeweils wieder eine Ebene höher zu kommen, befindet sich in der oberen linken Ecke ein roter \textit{Zurück}-Button.

Das Hauptmenü, sowie im Spiel das Pausenmenü, lassen sich sowohl mit der Maus, der Tastatur und dem Controller steuern.

\subsubsection*{Maus}
Um das Menü mit der Maus zu steuern muss der jeweils auszuwählende Button einfach mit der Maus ausgewählt und geklickt werden. Die Schieberegler in den Einstellungen können so ebenfalls mit der Maus gesteuert werden.

\subsubsection*{Tastatur}
Um mit der Tastatur durch die einzelnen Optionen zu navigieren können die Tasten \keys{W} und \keys{S}, sowie die Pfeiltasten \keys{\arrowkeyup} und \keys{\arrowkeydown} verwendet werden. Die aktuell ausgewählte Option ist dabei visuell hervorgehoben. Um eine Auswahl zu bestätigen muss die Enter Taste \keys{\return} gedrückt werden. Zum Verschieben der Schieberegler können zusätzlich die Pfeiltasten \keys{\arrowkeyleft} und \keys{\arrowkeyright} verwendet werden.

\subsubsection*{Controller}
Um mit dem Controller durch die einzelnen Optionen zu navigieren und die Schieberegler in den Einstellungen zu steuern können die Knöpfe des digitalen Steuerkreuzes, sowie die beiden analogen Steuersticks verwendet werden. Die aktuell ausgewählte Option ist dabei visuell hervorgehoben. Um eine Auswahl zu bestätigen muss die Bestätigungstaste des Controllers gedrückt werden. Bei einem XBox-Controller wäre dies die Taste \textbf{A}, bei einem Playstation-Controller die Taste \textbf{X}.

\section{Spiel}
\subsection{Ziel}
Das Ziel des Spiels ist es in der gegebenen Rundenzeit möglichst viele Items einzusammeln. Diese Items geben je nach Typ eine unterschiedliche Anzahl an Punkten. So gibt der \textbf{Apfel 1 Punkt}, die \textbf{Kirsche 2 Punkte} und die \textbf{Ananas 5 Punkte}. Der Spieler, der zum Ende der Runde die meisten Punkte hat, gewinnt. Sollte ein Punktegleichstand zwischen zwei oder mehreren Spielern herrschen, so wird der Spieler besser platziert, der während der Runde weniger oft gestorben ist.

\subsection{Steuerung}
Die Steuerung des Spieler-Charakter wird in den beiden nachfolgenden Abbildungen für Tastatur und Controller dargestellt. Als Beispiel für den Controller wird ein Playstation-5-Controller abgebildet, die Belegung ist aber auf jedem Controller gleich. Die Steuerung des, aus dem Spiel heraus aufrufbaren Pausenmenüs, ist bereits in Kapitel~\ref{menu} beschrieben.\\

\begin{figure}[H]
\centering
\includegraphics[width=125mm]{Figures/gamepad.jpg}
\end{figure}

\begin{figure}[H]
\centering
\includegraphics[width=105mm]{Figures/keyboard.jpg}
\end{figure}

\begin{bclogo}[logo=\bcattention, noborder=true, barre=none]{Wichtig!}
Möchte man mit mehreren Spielern im lokalen Split-Screen-Modus spielen, so ist die Anzahl der Spieler vor Levelbeginn im Hauptmenü festzulegen. Im Level selbst treten die Spieler dann durch Drücken der \textit{Spawn}-Taste auf ihrem jeweiligen Eingabegerät dem Spiel bei. Die Spieler-Charaktere lassen sich erst bewegen, sobald die angegebene Spieleranzahl erreicht ist.
\end{bclogo}