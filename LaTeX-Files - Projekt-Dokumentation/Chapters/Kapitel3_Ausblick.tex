\setlength{\parindent}{0em}

\chapter{Ausblick}

\label{Chapter3}

Für unseren Spiel-Prototyp war es uns wichtig alle von uns vorgesehenen Spielmechaniken zu implementieren und im ersten Level zu verwenden. Dies ist uns unserer Meinung nach auch gelungen. Dennoch gibt es natürlich einige Verbesserungsmöglichkeiten, Level-Erweiterungen und neue Spielmechaniken, welche in zukünftigen Versionen des Spiels umgesetzt werden können. Enige Ideen für die Zukunft sollen im Folgenden kurz beschrieben werden.

\subsubsection*{Weitere Levels}

Da wir bereits für das erste Level viele Spielmechaniken (Skripte, Prefabs) implementiert haben, welche für jedes weitere Level ebenfalls benötigt werden, sollten sich weitere Level deutlich schneller erstellen lassen als das erste. Selbst ohne neue Gegnertypen, Hindernisse oder Spielmechaniken liesen sich so neue kreative Herausforderungen für die Spieler erstellen.

\subsubsection*{Weitere Gegnertypen und Hindernisse}

Eine weitere Idee das Spiel in Zukunft um Inhalte zu erweitern, ist das Hinzufügen von neuen Gegnertypen und Hindernissen. Hierbei könnten die Gegnertypen beispielsweise andere Angriffsmuster oder Bewegungsmuster haben, als die bisherigen Gegner. Sowohl für weitere Gegner, als auch für weitere Hindernisse bieten die von uns verwendeten Unity-Assets \textit{Pixel Adventure 1} und \textit{Pixel Adventure 2} noch genügend Sprite-Sheets.

\subsubsection*{Neue Spielmechaniken}

Auch über neue Spielmechaniken, die man in Zukunft hinzufügen könnte, haben wir uns Gedanken gemacht und sind dabei auf folgende Ideen gekommen:

\begin{itemize}
    \item Spieler können sich gegenseitig beschießen und eliminieren und danach die Punkte des anderen klauen.
    \item Verschiebbare Kisten, mithilfe derer sich ansonsten zu hohe Ziele erreichen lassen und die Wege für andere Spieler blockieren lassen.
    \item Erweitern des Multiplayer-Modus, sodass Spieler auch über das Internet gegeneinander spielen können.
\end{itemize}

\subsubsection*{Zusätzliche Plattformen}

Zwar lässt sich das Spiel bereits auf Windows und MacOS über eine native Applikation spielen, sowie durch den WebGL-Build auf vielen weiteren internetfähigen Endgeräten, doch gibt es auch hier noch Erweiterungspotential. So könnte man das Spiel in Zukunft auch noch nativ für Konsolen umsetzen oder das Spiel als App für Smartphones umsetzen. In letzterem Fall wäre es noch praktisch die Steuerung des Spiels so zu erweitern, dass nicht mehr zwingend eine Tastatur oder ein Gamepad mit dem Smartphone verbunden werden muss. 
